\documentclass[12pt]{article}
\usepackage[UTF8]{ctex}
\usepackage{amsmath, amssymb, amsthm}  % 加载amsthm宏包
\usepackage{mathrsfs}
\usepackage{enumitem}
\usepackage{geometry}
\geometry{a4paper, left=2cm, right=2cm, top=2cm, bottom=2cm}

% 添加 hyperref 宏包以生成PDF书签
\usepackage{hyperref}
\hypersetup{
	colorlinks=true,      % 链接颜色
	linkcolor=blue,       % 内部链接颜色
	citecolor=blue,       % 引用颜色
	filecolor=magenta,    % 文件链接颜色
	urlcolor=cyan,        % URL链接颜色
	pdfborder={0 0 0},    % 去掉链接边框
	bookmarks=true,       % 生成PDF书签
	bookmarksnumbered=true % 书签显示章节编号
}

\newtheorem{theorem}{定理}  % 定义定理环境

\title{第三次课程讲义}
\author{谌秋辉}
\date{2025年4月8日笔记}

\begin{document}
	
	\maketitle
	
	\section{敛散性历史回顾}
	
	\subsection{主要发展阶段}
	\begin{enumerate}[leftmargin=2cm]
		\item \textbf{Fourier 的发现(1807年)}:首次提出 Fourier 级数理论,并认为其收敛性在物理问题中是自然成立的。
		
		\item \textbf{Dirichlet 的贡献(1829年)}:提出 Dirichlet 条件,证明了满足该条件的周期函数其 Fourier 级数逐点收敛。但当时普遍误认为所有连续函数的 Fourier 级数均收敛。
		
		\item \textbf{Du Bois-Reymond 的反例(1873年)}:构造了一个连续周期函数,其 Fourier 级数在某一点发散,这一结果颠覆了当时的认知。
		
		\item \textbf{Kolmogorov 的工作(1926年)}:证明存在 $L^1(\mathbb{T})$ 函数,其 Fourier 级数在每一点都发散。
		
		\item \textbf{Carleson 的突破(1966年)}:证明了 $L^2(\mathbb{T})$ 函数的 Fourier 级数几乎处处收敛:
		\[
		S_n(f,x) \rightarrow f(x), \quad \text{a.e.} \ x \in \mathbb{T}.
		\]
		
		\item \textbf{Hunt 的推广(1967年)}:将结果扩展至 $L^p(\mathbb{T})$ 空间($p>1$),证明对任意 $f \in L^p(\mathbb{T})$,有:
		\[
		S_n(f,x) \rightarrow f(x), \quad \text{a.e.} \ x \in \mathbb{T}.
		\]
	\end{enumerate}
	
	\subsection{总结与意义}
	\begin{itemize}
		\item Carleson 的证明极其复杂,但其结论(Carleson-Hunt 定理)奠定了现代 Fourier 分析的基石。
		\item 连续函数的 Fourier 级数收敛性问题并非完全解决,例如是否存在连续函数其 Fourier 级数在某个正测集上发散,仍是未解之谜。
		\item 这些成果深刻影响了调和分析、偏微分方程及信号处理等领域。
	\end{itemize}
	
	\section{各种求和理论}
	
	\subsection{Fejér 求和法(算术平均求和法)}
	定义 Fejér 和为:
	\[
	G_n(x) = \frac{1}{n+1} \sum_{k=0}^n S_k(f,x) = \frac{1}{n+1} \sum_{k=-n}^n \left( 1 - \frac{|k|}{n+1} \right) C_k e^{ikx}
	\]
	其积分形式可表示为:
	\[
	G_n(x) = \frac{1}{2\pi(n+1)} \int_{-\pi}^{\pi} f(x-t) \left( \frac{\sin\left( \frac{(n+1)t}{2} \right)}{\sin\left( \frac{t}{2} \right)} \right)^2 dt
	\]
	
	\begin{theorem}[Lebesgue]
		若 $f \in L^1(\mathbb{T})$,则:
		\begin{enumerate}[leftmargin=2cm]
			\item 若 $f$ 在 $x_0$ 处满足 Lebesgue 条件:
			\[
			\lim_{h \to 0} \frac{1}{h} \int_0^h |f(x_0 + t) - f(x_0)| dt = 0,
			\]
			则 $G_n(x_0) \to f(x_0) \ (n \to \infty)$.
			
			\item 在几乎处处意义下,$G_n(x) \to f(x) \ \text{a.e.} \ x \in \mathbb{T}$.
		\end{enumerate}
	\end{theorem}
	
	\begin{theorem}[$L^p$ 收敛性]
		若 $f \in L^p(\mathbb{T})$ 且 $1 \leq p < \infty$,则:
		\[
		\lim_{n \to \infty} \| f - G_n(f) \|_p = 0.
		\]
	\end{theorem}
	
	\subsection{Abel 求和与 Poisson 核}
	定义 Abel-Poisson 和为:
	\[
	F(r,x) = \sum_{k=-\infty}^{\infty} C_k r^{|k|} e^{ikx} = \frac{1}{2\pi} \int_{-\pi}^{\pi} f(t) P(r, x-t) dt,
	\]
	其中 Poisson 核为:
	\[
	P(r, \theta) = \frac{1 - r^2}{1 - 2r\cos\theta + r^2} = \sum_{n=-\infty}^\infty r^{|n|} e^{in\theta}.
	\]
	
	\begin{itemize}
		\item \textbf{解析性}:当 $r \to 1^-$ 时,$F(r,x)$ 在单位圆内解析,边界值与原函数相关。
		\item \textbf{收敛性}:若 $f \in L^p(\mathbb{T}) \ (p \geq 1)$,则 $F(r,x) \to f(x) \ \text{a.e.}$ 当 $r \to 1^-$。
	\end{itemize}
	
	\section{Fourier级数逐点收敛}
	
	\subsection{基本定义与部分和}
	设 \( f \in L^1[-\pi, \pi] \) 为广义周期函数,其 Fourier 部分和为:
	\[
	S_n(f, x) = \frac{a_0}{2} + \sum_{k=1}^n \left( a_k \cos kx + b_k \sin kx \right),
	\]
	其中系数由积分表达式给出:
	\[
	a_k = \frac{1}{\pi} \int_{-\pi}^{\pi} f(u) \cos(ku) \, du, \quad 
	b_k = \frac{1}{\pi} \int_{-\pi}^{\pi} f(u) \sin(ku) \, du.
	\]
	
	\subsection{积分表示与 Dirichlet 核}
	部分和可重写为积分形式:
	\[
	S_n(f, x) = \frac{1}{2\pi} \int_{-\pi}^{\pi} f(u) \, du + \frac{1}{\pi} \sum_{k=1}^n \int_{-\pi}^{\pi} f(u) \cos k(u-x) \, du.
	\]
	进一步化简为:
	\[
	S_n(f, x) = \frac{1}{\pi} \int_{-\pi}^{\pi} f(u) \left( \frac{1}{2} + \sum_{k=1}^n \cos k(u-x) \right) du.
	\]
	利用三角恒等式:
	\[
	\frac{1}{2} + \sum_{k=1}^n \cos k\theta = \frac{\sin\left( (n+\frac{1}{2})\theta \right)}{2 \sin \frac{\theta}{2}},
	\]
	可得 Dirichlet 核表示:
	\[
	S_n(f, x) = \frac{1}{\pi} \int_{-\pi}^{\pi} f(u) \cdot \frac{\sin\left( (n+\frac{1}{2})(u-x) \right)}{2 \sin \frac{u-x}{2}} \, du.
	\]
	
	\subsection{Dirichlet 核的性质}
	定义 Dirichlet 核:
	\[
	D_n(\theta) = \frac{1}{2\pi} \cdot \frac{\sin\left( (n+\frac{1}{2})\theta \right)}{\sin \frac{\theta}{2}},
	\]
	则部分和可进一步简化为:
	\[
	S_n(f, x) = \int_{-\pi}^{\pi} f(u) D_n(u-x) \, du.
	\]
	
	\subsection{收敛性分析}
	\begin{itemize}
		\item \textbf{逐点收敛}:若 \( f \) 在 \( x \) 处满足 Dini 条件,即
		\[
		\int_{-\pi}^{\pi} \left| \frac{f(x+t) - f(x)}{t} \right| dt < \infty,
		\]
		则 \( S_n(f, x) \to f(x) \) 当 \( n \to \infty \)。
		
		\item \textbf{均方收敛}:对 \( f \in L^2[-\pi, \pi] \),有
		\[
		\lim_{n \to \infty} \| S_n(f) - f \|_{L^2} = 0.
		\]
	\end{itemize}
	
	\section{Fourier级数的复数形式与Dirichlet核}
	
	\subsection{复数形式部分和}
	周期函数$f \in L^1(\mathbb{T})$的Fourier部分和可表示为:
	\[
	S_n(f, x) = \sum_{k=-n}^n C_k e^{ikx} = \frac{1}{2\pi} \int_{-\pi}^{\pi} f(u) \sum_{k=-n}^n e^{ik(x-u)} du,
	\]
	其中复数系数为:
	\[
	C_k = \frac{1}{2\pi} \int_{-\pi}^{\pi} f(u) e^{-iku} du.
	\]
	
	\subsection{Dirichlet核的积分表示}
	通过几何级数求和公式,部分和可化简为卷积形式:
	\[
	S_n(f, x) = \int_{-\pi}^{\pi} f(u) D_n(x-u) du = (f * D_n)(x),
	\]
	其中Dirichlet核定义为:
	\[
	D_n(t) = \frac{1}{2\pi} \sum_{k=-n}^n e^{ikt} = \frac{\sin\left( (n+\frac{1}{2})t \right)}{2\pi \sin\left( \frac{t}{2} \right)}.
	\]
	
	\subsection{Dirichlet核的性质}
	\begin{itemize}
		\item \textbf{归一性}:
		\[
		\int_{-\pi}^{\pi} D_n(t) dt = 1.
		\]
		
		\item \textbf{对称性}:
		\[
		D_n(-t) = D_n(t).
		\]
		
		\item \textbf{正交投影}:$S_n$是从$L^2(\mathbb{T})$到$2n+1$维三角多项式空间的投影算子,即
		\[
		S_n: L^2(\mathbb{T}) \to \mathcal{T}_n, \quad \mathcal{T}_n = \mathrm{span}\{ e^{ikx} \}_{k=-n}^n.
		\]
	\end{itemize}
	
	\subsection{收敛性评注}
	Dirichlet核的振荡特性导致:
	\begin{itemize}
		\item 对$f \in L^1(\mathbb{T})$,$S_n(f,x)$的逐点收敛需额外条件(如Dini条件)。
		\item 对$f \in L^2(\mathbb{T})$,部分和在$L^2$范数下收敛:
		\[
		\lim_{n \to \infty} \| S_n(f) - f \|_{L^2} = 0.
		\]
	\end{itemize}
	
	\section{fourier级数的收敛性分析}
	
	\subsection{基本假设与收敛目标}
	设 $f \in L^2(\mathbb{T})$ 为周期函数,研究其 Fourier 部分和 $S_n(f,x)$ 在点 $x \in \mathbb{R}$ 处的收敛性。  
	假设存在常数 $c \in \mathbb{C}$ 使得当 $n \to \infty$ 时:
	\[
	S_n(f, x) \to c.
	\]
	
	\subsection{偏差积分表达式}

	\[
	c = \int_{-\pi}^{\pi} c \cdot D_n(t) \, dt
	\]
	这是因为Dirichlet核满足归一性:
	\[
	\int_{-\pi}^{\pi} D_n(t) \, dt = 1.
	\]
	其中 $D_n(t)$ 为 Dirichlet 核:
	\[
	D_n(t) = \frac{1}{2\pi} \cdot \frac{\sin\left( (n+\frac{1}{2})t \right)}{\sin\left( \frac{t}{2} \right)}.
	\]
	积分运算的线性性允许将偏差分解为:
	\[
	\begin{aligned}
		S_n(f, x) - c &= \int_{-\pi}^{\pi} f(x-t) D_n(t) \, dt - \int_{-\pi}^{\pi} c D_n(t) \, dt \\
		&= \int_{-\pi}^{\pi} \left[ f(x-t) - c \right] D_n(t) \, dt.
	\end{aligned}
	\]

	
	
	\subsection{积分分解与对称性}
	将积分分解为对称部分:
	\[
	S_n(f, x) - c = \int_{0}^{\pi} \left( f(x-t) - c \right) D_n(t) \, dt + \int_{-\pi}^{0} \left( f(x-t) - c \right) D_n(t) \, dt.
	\]
	利用 Dirichlet 核的偶函数性质 $D_n(-t) = D_n(t)$,可将第二项变量替换 $t \to -t$:
	\[
	\int_{-\pi}^{0} \left( f(x-t) - c \right) D_n(t) \, dt = \int_{0}^{\pi} \left( f(x+t) - c \right) D_n(t) \, dt.
	\]
	因此,总偏差可合并为:
	\[
	S_n(f, x) - c = \int_{0}^{\pi} \left[ f(x-t) + f(x+t) - 2c \right] D_n(t) \, dt.
	\]
	
	\subsection{收敛性条件}
	\begin{enumerate}[leftmargin=2cm]
		\item \textbf{Dini 条件}:若存在 $\delta > 0$ 使得
		\[
		\int_{0}^{\delta} \frac{|f(x+t) + f(x-t) - 2c|}{t} \, dt < \infty,
		\]
		则 $S_n(f, x) \to c$ 当 $n \to \infty$。
		
		\item \textbf{Jordan 条件}:若 $f$ 在 $x$ 处有界变差,则 $c = \frac{f(x^+) + f(x^-)}{2}$ 且收敛成立。
	\end{enumerate}
	
	\subsubsection{注记}
	\begin{itemize}
		\item 当 $c = f(x)$ 时,收敛性要求 $f$ 在 $x$ 处满足某种正则性(如连续性或 Dini 条件)。
		\item 对 $f \in L^2(\mathbb{T})$,部分和在 $L^2$ 范数下收敛:
		\[
		\lim_{n \to \infty} \| S_n(f) - f \|_{L^2} = 0.
		\]
	\end{itemize}
	
	\subsection{积分化简}
	设 \(\psi_x(t) = f(x+t) + f(x-t) - 2c\),且 \(\psi_x(-t) = \psi_x(t)\)。假设 \(\psi_x(t) \in L^1(0, \pi)\),则有:
	\[
	\int_0^\pi \psi_x(t) \, dt = \int_0^\pi f(x+t) \, dt + \int_0^\pi f(x-t) \, dt - 2c\pi.
	\]
	
	部分和 \(S_n(f, x) - c\) 可表示为:
	\[
	S_n(f, x) - c = \int_0^\pi \psi_x(t) D_n(t) \, dt,
	\]
	其中 Dirichlet 核为:
	\[
	D_n(t) = \frac{\sin\left( (n+\frac{1}{2})t \right)}{2\pi \sin \frac{t}{2}}.
	\]
	
	\subsection{积分化简与Riemann-Lebesgue引理}
	
	将积分化简为:
	\[
	S_n(f, x) - c = \int_0^\pi \psi_x(t) \cdot \frac{\sin\left( (n+\frac{1}{2})t \right)}{2\pi \sin \frac{t}{2}} \, dt.
	\]
	
	进一步分解为:
	\[
	= \int_0^\pi \psi_x(t) \cdot \frac{\sin(nt) \cos \frac{t}{2} + \cos(nt) \sin \frac{t}{2}}{2\pi \sin \frac{t}{2}} \, dt.
	\]
	
	化简后得到:
	\[
	= \int_0^\pi \psi_x(t) \cdot \frac{\cos \frac{t}{2} \sin nt}{2\pi \sin \frac{t}{2}} \, dt + \frac{1}{2\pi} \int_0^\pi \psi_x(t) \cos(nt) \, dt.
	\]
	
	应用Riemann-Lebesgue引理,当 \(n \to \infty\) 时:
	\[
	= \frac{1}{2\pi} \int_0^\pi \psi_x(t) \cos \frac{t}{2} \cdot \frac{\sin nt}{\sin \frac{t}{2}} \, dt + o(1).
	\]
	
	\subsubsection{注记}
	
	对于 \(\psi_x(t) \cos \frac{t}{2}\),不能直接应用Riemann-Lebesgue引理,因为 \(\frac{t}{2}\) 在0点有奇异性,不属于 \(L^1(0, \pi)\)。
	
	
	\subsection{偏差积分表达式与Dirichlet核}
	
	设 \(\delta \in (0, \frac{\pi}{2})\),部分和 \(S_n(f, x) - c\) 可表示为:
	\[
	S_n(f, x) - c = \frac{1}{2\pi} \int_\delta^\pi \psi_x(t) \cdot \frac{\cos \frac{t}{2} \sin nt}{\sin \frac{t}{2}} \, dt + \frac{1}{2\pi} \int_0^\delta \psi_x(t) \cdot \frac{\cos \frac{t}{2} \sin nt}{\sin \frac{t}{2}} \, dt + o(1).
	\]
	
	其中 \(\psi_x(t) = f(x+t) + f(x-t) - 2c\)。
	
	\subsection{应用Riemann-Lebesgue引理}
	
	对于第一项:
	\[
	\frac{1}{2\pi} \int_\delta^\pi \psi_x(t) \cdot \frac{\cos \frac{t}{2} \sin nt}{\sin \frac{t}{2}} \, dt = o(1),
	\]
	因为 \(\psi_x(t) \cdot \frac{\cos \frac{t}{2}}{\sin \frac{t}{2}} \in L^1([\delta, \pi])\),应用Riemann-Lebesgue引理。
	

	对于第二项:
	\[
	\frac{1}{2\pi} \int_0^\delta \psi_x(t) \cdot \frac{\cos \frac{t}{2} \sin nt}{\sin \frac{t}{2}} \, dt,
	\]
	需要进一步分析。
	
	\subsection{归纳与结论}
	
	归纳:
	\[
	S_n(f, x) \to c \quad (n \to \infty) \quad \Leftrightarrow \quad \int_0^\delta \psi_x(t) \cdot \frac{\cos \frac{t}{2}}{\sin \frac{t}{2}} \sin nt \, dt \to 0 \quad (n \to \infty).
	\]
	
	进一步分解:

	\[
	\int_0^\delta \psi_x(t) \cdot \cos\left( \frac{t}{2} \right) \cdot \frac{t}{\sin\left( \frac{t}{2} \right)} \cdot \frac{1}{t} \sin(nt) \, dt \to 0 \quad (n \to \infty).
	\]
	

	\subsubsection{注记}
	
	\(f\) 在 \((0, \delta)\) 的取值有关,任意小邻域的取值决定收敛性。
	
	\section{Dini 判别法}
	\subsection{Dini条件}
	
		\begin{theorem}[Dini 判别法]
		在Dini条件下,Fourier 部分和满足:
		\[
		\lim_{n \to \infty} S_n(f, x) = C.
		\]
	\end{theorem}
 $f \in L^1(\mathbb{T})$,$C \in \mathbb{C}$ 为常数。偏差函数:
	\[
	\psi_x(t) = f(x+t) + f(x-t) - 2C,
	\]
	存在 $\delta > 0$ 使得:
	\[
	\int_0^\delta \frac{|\psi_x(t)|}{t} \, dt < \infty.
	\]


	
	\subsection{证明}
利用 Dirichlet 核的对称性,部分和偏差可表示为:
		\[
		S_n(f, x) - C = \int_0^\delta \psi_x(t) \cdot \frac{\sin\left( (n+\frac{1}{2})t \right)}{2\pi \sin\left( \frac{t}{2} \right)} \, dt + o(1).
		\]		
将 Dirichlet 核拆分为主导项与余项:
	\[
 \frac{1}{2\pi} \int_0^\delta \frac{\psi_x(t)}{t} \cdot \frac{t \cos\left( \frac{t}{2} \right)}{\sin\left( \frac{t}{2} \right)} \cdot \sin(nt) \, dt+ o(1)
	\]
		从而偏差积分可近似为:
		\[
		S_n(f, x) - C =  \frac{1}{2\pi} \int_0^\delta \frac{\psi_x(t)}{t} \cdot \frac{t \cos\left( \frac{t}{2} \right)}{\sin\left( \frac{t}{2} \right)} \cdot \sin(nt) \, dt+ o(1).
		\]
应用Riemann-Lebesgue 引理
由于 $\frac{\psi_x(t)}{t} \in L^1(0, \delta)$ 且 $\frac{t \cdot \cos\left( \frac{t}{2} \right)}{\sin\left( \frac{t}{2} \right)}$ 在 $(0,\delta)$ 上连续,则
	\[
	\frac{1}{2\pi} \int_0^\delta \frac{\psi_x(t)}{t} \cdot \frac{t \cos\left( \frac{t}{2} \right)}{\sin\left( \frac{t}{2} \right)} \sin(nt) \, dt \to 0 \quad (n \to \infty).
	\]
		
		因此 $S_n(f, x) \to C$ 当 $n \to \infty$。

	
	\subsubsection{注记}
	\begin{itemize}
		\item Dini 条件的物理意义:偏差函数 $\psi_x(t)$ 在 $t \to 0$ 时的衰减速度快于 $1/t$。
		\item 当 $C = f(x)$ 时,条件退化为经典 Dini 条件:
		\[
		\int_0^\delta \frac{|f(x+t) + f(x-t) - 2f(x)|}{t} \, dt < \infty.
		\]
	\end{itemize}
	
	\subsection{推论1:可微性条件下的收敛性}
	
	设 \( f \in L(-\pi, \pi) \),若 \( f \) 在 \( x \) 处可微,则 \( S_n(f, x) \to f(x) \) 当 \( n \to \infty \)。
	
证明
	在Dini定理中,令 \( c = f(x) \)。
	
	定义:
	\[
	\psi_x(t) = f(x+t) + f(x-t) - 2f(x).
	\]
	
	则:
	\[
	\frac{\psi_x(t)}{t} = \frac{f(x+t) - f(x)}{t} + \frac{f(x-t) - f(x)}{-t}.
	\]
	
	进一步化简为:
	\[
	= \frac{f(x+t) - f(x)}{t} - \frac{f(x-t) - f(x)}{t}.
	\]
	
	当 \( t \to 0 \) 时:
	\[
	\lim_{t \to 0} \frac{\psi_x(t)}{t} = f'(x) - f'(x) = 0.
	\]
	
	因此,若:
	\[
	\int_0^\delta \frac{|\psi_x(t)|}{t} \, dt < \infty,
	\]
	则 \( S_n(f, x) \to f(x) \) 当 \( n \to \infty \)。
	
	\subsubsection{注记}
	
	对任意 \(\varepsilon > 0\),存在 \(\delta_1 > 0\),当 \( t \in (0, \delta_1) \) 时:
	\[
	\left| \frac{\psi_x(t)}{t} \right| < \varepsilon.
	\]
	
	因此:
	\[
	\int_0^\delta \frac{|\psi_x(t)|}{t} \, dt = \int_0^{\delta_1} \frac{|\psi_x(t)|}{t} \, dt + \int_{\delta_1}^\delta \frac{|\psi_x(t)|}{t} \, dt.
	\]
	
	估计每一项:
	\[
	\leq \varepsilon \delta_1 + \int_{\delta_1}^\delta \frac{|\psi_x(t)|}{t} \, dt.
	\]
	
	由于 \(\frac{1}{t}\) 在 \((\delta_1, \delta)\) 上是连续的,积分收敛。
	
\subsubsection{结论}
	
	因此:
	\[
	S_n(f, x) \to f(x) \quad (n \to \infty).
	\]
	
	\subsection{推论2:Lipschitz条件下的收敛性}
	
	设 \( f \in L(-\pi, \pi) \),若 \( f \) 在 \( x \) 处满足Lipschitz条件,即存在常数 \( M > 0 \) 和 \( \alpha \in (0, 1] \),使得:
	\[
	|f(x+t) - f(x)| \leq M |t|^\alpha \quad  \alpha \in (0, 1].
	\]
	
	则 \( S_n(f, x) \to f(x) \) 当 \( n \to \infty \)。
	
证明,应用Dini定理

	\[
	\psi_x(t) = f(x+t) + f(x-t) - 2f(x).
	\]
	
	则:
	\[
	\left| \frac{\psi_x(t)}{t} \right| \leq \left| \frac{f(x+t) - f(x)}{t} \right| + \left| \frac{f(x-t) - f(x)}{t} \right| \leq 2M |t|^{\alpha - 1}.
	\]
	
	因此,积分:
	\[
	\int_0^\delta \frac{|\psi_x(t)|}{t} \, dt \leq 2M \int_0^\delta |t|^{\alpha - 1} \, dt < \infty.
	\]
	
	\subsubsection{结论}
	
	由Dini定理可知:
	\[
	S_n(f, x) \to f(x) \quad (n \to \infty).
	\]
	

	\subsection{处理Dini条件的积分转换}
	
	在处理Dini条件时,将积分:

\[
\int_0^\delta \psi_x(t) \cot \frac{t}{2} \sin nt \, dt
\]
转换为:
\[
\int_0^\delta \frac{\psi_x(t)}{t} \sin nt \, dt
\]
	
	是基于以下分析,定义函数 \( g(t) \) 如下:
	\[
	g(t) = 
	\begin{cases} 
		\cot \frac{t}{2} - \frac{1}{t}, & t \in [-\pi, \pi] \setminus \{0\} \\
		0, & t = 0
	\end{cases}
	\]
	在 \([- \pi, \pi]\) 上连续。
	
	\subsubsection{极限分析}
	
	计算极限:
\[
\lim_{t \to 0} \left( \alpha \cot \frac{t}{2} - \frac{1}{t} \right).
\]

其中 \(\alpha = \frac{1}{2}\),则:
\[
\lim_{t \to 0} \left( \frac{1}{2} \cot \frac{t}{2} - \frac{1}{t} \right).
\]

展开 \(\cot \frac{t}{2}\):
\[
\cot \frac{t}{2} = \frac{\cos \frac{t}{2}}{\sin \frac{t}{2}}.
\]

当 \( t \to 0 \) 时,利用等价无穷小:
\[
\sin \frac{t}{2} \sim \frac{t}{2}, \quad \cos \frac{t}{2} \sim 1 - \frac{t^2}{8}.
\]

因此:
\[
\frac{1}{2} \cot \frac{t}{2} \sim \frac{1}{2} \cdot \frac{1 - \frac{t^2}{8}}{\frac{t}{2}} = \frac{1}{2} \cdot \frac{2}{t} \left( 1 - \frac{t^2}{8} \right) = \frac{1}{t} \left( 1 - \frac{t^2}{8} \right).
\]

所以:
\[
\frac{1}{2} \cot \frac{t}{2} - \frac{1}{t} \sim \frac{1}{t} \left( 1 - \frac{t^2}{8} \right) - \frac{1}{t} = -\frac{t}{8}.
\]

当 \( t \to 0 \) 时:
\[
\lim_{t \to 0} \left( \frac{1}{2} \cot \frac{t}{2} - \frac{1}{t} \right) = 0.
\]

\subsubsection{结论}

因此,\( g(t) \) 在 \( t = 0 \) 处连续,且:
\[
\int_0^\delta \psi_x(t) \cot \frac{t}{2} \sin nt \, dt = \int_0^\delta \frac{\psi_x(t)}{t} \sin nt \, dt.
\]
	
	\subsection{积分性质与应用}
	
	
	
	\subsubsection{积分第二中值定理}
	设 \( f \) 在 \([a, b]\) 上Riemann可积。
	
	若 \( g \) 在 \([a, b]\) 上递减且 \( g(x) \geq 0 \),则存在 \(\xi \in [a, b]\),使得:
		\[
		\int_a^b f(x) g(x) \, dx = g(a) \int_a^\xi f(x) \, dx.
		\]
  
  若 \( g \) 在 \([a, b]\) 上递增且 \( g(x) \geq 0 \),则存在 \(\eta \in [a, b]\),使得:
		\[
		\int_a^b f(x) g(x) \, dx = g(b) \int_\eta^b f(x) \, dx.
		\]

	
	\subsubsection{一般情况}
	
	若 \( f \) 在 \([a, b]\) 上Riemann可积,且 \( g \) 单调,则存在 \(\xi, \eta \in [a, b]\),使得:
	\[
	\int_a^b f(x) g(x) \, dx = g(a) \int_a^\xi f(x) \, dx + g(b) \int_\eta^b f(x) \, dx.
	\]
	
	\subsubsection{注记}
	这种推论可以构造很多条件,只需要保证存在 $\delta > 0$ 使得
\[
\int_{0}^{\delta} \frac{\psi_x(t)}{t} \, dt < \infty \quad 
\]

如 \( f \in C^2[-\pi, \pi] \) 且满足 Hölder 条件:
\[
|f(x+\epsilon) - f(x)| \leq M \cdot \frac{1}{\left|\ln|x|\right|^{\epsilon+1}} \quad (\epsilon > 0)
\]

则 
\[
S_n(f, x) \to f(x) \quad (n \to \infty).
\]
	即 Fourier 级数在点 $x$ 处收敛于 $f(x)$。
	
\section{Jordan判别法}

\begin{theorem}[Jordan判别法]
	设 $f \in L^1[-\pi, \pi]$,且 $f$ 在点 $x$ 的某邻域 $U(x,r)$ 上有界变差,则其Fourier级数部分和满足:
	\[
	\lim_{n \to \infty} S_n(f, x) = \frac{1}{2} \left[ f(x+0) + f(x-0) \right].
	\]
\end{theorem}

\subsection{证明思路}
\begin{enumerate}[leftmargin=2cm]
	\item \textbf{积分估计引理}:对任意 $a,b \in \mathbb{R}$,有
	\[
	\left| \int_a^b \frac{\sin x}{x} dx \right| \leq 6.
	\]
	
	\begin{proof}
		分情况讨论:
		\begin{itemize}
			\item 当 $1 \leq |a| \leq b$ 时,由第二积分中值定理,存在 $\xi \in [a,b]$ 使得:
			\[
			\left| \int_a^b \frac{\sin t}{t} dt \right| = \frac{1}{|a|} \left| \int_a^\xi \sin t dt \right| \leq \frac{2}{|a|} \leq 2.
			\]
			
			\item 当 $0 \leq a \leq b \leq 1$ 时:
			\[
			\left| \int_a^b \frac{\sin t}{t} dt \right| \leq \int_a^b 1 dt = b - a \leq 1.
			\]
			
			\item 综合两种情况可得:
			\[
			\left| \int_a^b \frac{\sin x}{x} dx \right| \leq 3 \quad \text{(同号积分)}.
			\]
			
			\item 对于一般情况($a<0<b$):
			\[
			\left| \int_a^b \frac{\sin x}{x} dx \right| \leq \int_0^{|a|} + \int_0^b \leq 6.
			\]
		\end{itemize}
	\end{proof}
\end{enumerate}

\subsubsection{应用说明}
\begin{itemize}
	\item 有界变差函数类包含分段单调函数,故Jordan判别法比Dini条件适用范围更广。
	\item 当$f$在$x$处连续时,收敛值为$f(x)$;在跳跃间断点收敛于左右极限的平均值。
\end{itemize}
	
	\subsection{Jordan判别法证明步骤}
	设 \( f \in BV(x-r, x+r) \),则 \( f(x+0) \) 和 \( f(x-0) \) 存在(因有界变差函数在每点存在单侧极限)。令:
	\[
	c = \frac{1}{2} \left( f(x+0) + f(x-0) \right),
	\]
	并定义偏差函数:
	\[
	\psi_x(t) = f(x+t) + f(x-t) - f(x+0) - f(x-0).
	\]
	
	\subsubsection{Jordan分解定理应用}
	根据Jordan分解定理,存在单调递增函数 \( h_1, h_2 \) 使得:
	\[
	\psi_x(t) = h_1(t) - h_2(t), \quad \text{且} \quad h_1(0+) = h_2(0+) = 0.
	\]
	
	\subsubsection{偏差积分分解}
	部分和偏差可表示为:
	\[
	S_n(f, x) - c = \frac{1}{2\pi} \int_0^\delta \psi_x(t) \cdot \frac{\sin\left( (n+\frac{1}{2})t \right)}{\sin\left( \frac{t}{2} \right)} dt + o(1).
	\]
	
	\subsubsection{积分估计}
	对任意 \( \epsilon > 0 \),存在 \( \delta > 0 \) 使得:
\[
\begin{aligned}
	S_n(f, x) - C 
	&= S_n(f, x) - \frac{1}{2} \left[ f(x+0) + f(x-0) \right] \\
	&= -\frac{1}{\pi} \int_0^\delta \psi_x(t) \frac{\sin(nt)}{t} \, dt + o(1)
\end{aligned}
\]
根据积分第二中值定理,
\[
\begin{aligned}
	\left| S_n(f, x) - C \right| 
	&\leq \left| \frac{1}{\pi} \int_0^\delta h_1(t) \frac{\sin(nt)}{t} \, dt \right| 
	+ \left| \frac{1}{\pi} \int_0^\delta h_2(t) \frac{\sin(nt)}{t} \, dt \right| 
	+ \epsilon \\
	&= \left| \frac{1}{\pi} h_1(\delta) \int_{\xi_1}^{\delta } \frac{\sin(nt)}{t} \, dt \right| 
	+ \left| \frac{1}{\pi} h_2(\delta) \int_{\xi_2}^{\delta} \frac{\sin(nt)}{t} \, dt \right| 
	+ \epsilon \\
	&\leq \left( \frac{6+6}{\pi} \right) \epsilon + \epsilon 
	= \left( 1 + \frac{12}{\pi} \right)\epsilon.
\end{aligned}
\]

\section{数学分析中的经典收敛定理}

\begin{theorem}[Dirichlet收敛定理]
	设函数 $f$ 在区间 $[-\pi, \pi]$ 上满足:
	\begin{itemize}
		\item 分段光滑
		\item 在 $[-\pi, \pi]$ 上除有限个第一类间断点外连续
	\end{itemize}
	则对任意 $x \in (-\pi, \pi)$,其Fourier级数部分和满足:
	\[
	S_n(x) \to \frac{f(x+0) + f(x-0)}{2} \quad (n \to \infty).
	\]
\end{theorem}

\subsection{收敛性证明}
定义偏差积分:
\[
\int_0^\delta \frac{\psi_x(t)}{t} \sin(nt) \, dt = \int_0^\delta \frac{f(x+t)-f(x+0)}{t} \sin(nt) \, dt + \int_0^\delta \frac{f(x-t)-f(x-0)}{t} \sin(nt) \, dt
\]
其中定义辅助函数:
\[
\begin{aligned}
	\varphi_1(t) = \frac{f(x+t)-f(x+0)}{t}, \quad t \in (0, \delta] \\
	\varphi_2(t) = \frac{f(x-t)-f(x-0)}{t}, \quad t \in (0, \delta]
\end{aligned}
\]

\subsubsection{正则性分析}
\begin{itemize}
	\item 在 $t \to 0^+$ 时(分段光滑保证左右导数存在)
	\[
	\varphi_1(0^+) = \lim_{t \to 0^+} \frac{f(x+t)-f(x+0)}{t} = f'(x^+)
	\]
	\[
	\varphi_2(0^+) = \lim_{t \to 0^+} \frac{f(x-t)-f(x-0)}{t} = -f'(x^-)
	\]
	\item $\varphi_1, \varphi_2$ 在 $[0, \delta]$ 上连续
\end{itemize}

应用Riemann-Lebesgue引理
\[
\int_0^\delta \varphi_i(t) \sin(nt) \, dt = o(1) \quad (n \to \infty),\quad i=1,2
\]
因此原积分满足:
\[
\int_0^\delta \frac{\psi_x(t)}{t} \sin(nt) \, dt = o(1) \quad (n \to \infty).
\]


	
\end{document}